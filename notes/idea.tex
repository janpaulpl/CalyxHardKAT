\documentclass[sigplan,10pt,authorversion,nonacm]{acmart}
% \documentclass[sigplan,10pt,anonymous,review]{acmart}
\settopmatter{printfolios=false,printccs=false,printacmref=false}

% Required packages
\usepackage{amsmath}            % For math environments
\usepackage{amsthm}             % For theorem environments
\usepackage{mathpartir}
\usepackage{listings}           % For code listings
\usepackage{color}              % For defining custom colors
\usepackage{xcolor}             % Required for setting colors in listings
\usepackage{caption}
\usepackage{tikz}               % For diagrams
\usetikzlibrary{positioning, arrows.meta, shapes.geometric, fit}
\usepackage{syntax}  % For drawing network diagrams
\usepackage{proof}              % For tableaux-style proofs
\usepackage{adjustbox}
\usepackage{mdframed}

% \acmConference[CPP'25]{Certified Programs and Proofs}{January 20--21, 2025}{Denver, USA}

% New command for centering figures
\newcommand{\centeredImage}[2][]{%
  \begin{center}
    \includegraphics[#1]{#2}
  \end{center}
}
% If \coloneqq causes issues, redefine it here
\newcommand{\coloneqq}{\mathrel{:=}}

\theoremstyle{remark}
\newtheorem{remark}{Remark}

% Listing settings for OCaml
\lstdefinelanguage{Calyx}{
  keywords={let, in, type, match, with, function, fun, rec, struct, end, open, module, and, if, then, else, for, while, to, do, done, begin, of},
  keywordstyle=\bfseries\color{blue},
  comment=[l]{(*},
  commentstyle=\color{gray}\ttfamily,
  stringstyle=\color{red},
  morestring=[b]",
  moredelim=[s][\color{gray}]{(*}{*)},
  literate={->}{{$\rightarrow$}}1 {=>}{{$\Rightarrow$}}1 {<-}{{$\leftarrow$}}1,
}

% General settings for listings
\lstset{
  basicstyle=\ttfamily\footnotesize,
  columns=fullflexible,
  keepspaces=true,
  breaklines=true,
  frame=tb,                    % Adds a frame above and below the code block
  numbers=left,                % Line numbers on the left
  numberstyle=\tiny\color{gray}, % Line number style
  stepnumber=1,                % Number every line
  numbersep=5pt,               % Space between line numbers and code
  showstringspaces=false,
  tabsize=2,
  captionpos=b,
  framextopmargin=2pt,
  framexbottommargin=2pt,
  backgroundcolor=\color{white},
  xleftmargin=5pt,
}

\title{Proposal: Concurrent KAT for Calyx}

\author{Jan-Paul Ramos-D\'{a}vila}
\affiliation{%
  \institution{Cornell University}
  \city{Ithaca}
  \state{NY}
  \country{USA}}
\email{jvr34@cornell.edu}
\orcid{0000-0003-1055-6785}

\renewcommand{\shortauthors}{Ramos-D\'{a}vila}

\begin{document}

\begin{abstract}
This proposal outlines a project to formalize and implement parallelism semantics in Calyx using ideas from Kleene Algebra with Tests (KAT) and Synchronous Kleene Algebra (SKA). The project aims to provide a rigorous foundation for reasoning about parallel execution in Calyx while addressing its specific needs for hardware design.
\end{abstract}

\maketitle

\section{Project Outline}

\subsection{Background and Motivation}
\begin{itemize}
    \item Overview of Calyx and its current \texttt{par} semantics
    \item Need for formal semantics to reason about parallel execution
    \item Introduction to Kleene Algebra with Tests and Synchronous Kleene Algebra
\end{itemize}

\subsection{Formalization of Calyx Parallelism using SKA}
\begin{itemize}
    \item Define a core subset of Calyx focusing on \texttt{par} constructs
    \item Map Calyx constructs to SKA operators
    \item Define semantics for \texttt{par} using SKA's synchronous product operator
    \item Prove key properties of \texttt{par} using SKA axioms
\end{itemize}

\subsection{Extend SKA for Calyx-specific Needs}
\begin{itemize}
    \item Introduce latency-insensitive semantics to SKA
    \item Define a "loose" synchronous product for non-deterministic execution order
    \item Prove soundness and completeness of the extended algebra
\end{itemize}

\subsection{Operational Semantics and Implementation}
\begin{itemize}
    \item Define small-step operational semantics for Calyx \texttt{par} based on the SKA model
    \item Implement a reference interpreter for the core Calyx subset
    \item Develop compilation strategies for \texttt{par} based on the formal semantics
\end{itemize}

\section{Operational Semantics}

The operational semantics for Calyx's par construct can be defined using a small-step semantics approach. This aligns most closely with the "Explicit Synchronization" semantics, as it allows for explicit control over the execution order and synchronization points.

We define the following judgment for the execution of Calyx programs:

\begin{mathpar}
\inferrule*[right=Step]{
  \langle p, \sigma \rangle \rightarrow \langle p', \sigma' \rangle
}{
  \text{Program $p$ in state $\sigma$ steps to $p'$ with new state $\sigma'$}
}
\end{mathpar}

For the par construct, we define the following rules:

\begin{mathpar}
\inferrule*[right=Par-Left]{
  \langle p_1, \sigma \rangle \rightarrow \langle p_1', \sigma' \rangle
}{
  \langle \texttt{par}\{p_1, p_2\}, \sigma \rangle \rightarrow \langle \texttt{par}\{p_1', p_2\}, \sigma' \rangle
}

\inferrule*[right=Par-Right]{
  \langle p_2, \sigma \rangle \rightarrow \langle p_2', \sigma' \rangle
}{
  \langle \texttt{par}\{p_1, p_2\}, \sigma \rangle \rightarrow \langle \texttt{par}\{p_1, p_2'\}, \sigma' \rangle
}

\inferrule*[right=Par-Done]{
  p_1 = \texttt{done} \quad p_2 = \texttt{done}
}{
  \langle \texttt{par}\{p_1, p_2\}, \sigma \rangle \rightarrow \langle \texttt{done}, \sigma \rangle
}
\end{mathpar}

For groups, we can define:

\begin{mathpar}
\inferrule*[right=Group-Start]{
  g \text{ is a group}
}{
  \langle g, \sigma \rangle \rightarrow \langle g_\text{body}, \sigma[g.\texttt{go} \mapsto 1] \rangle
}

\inferrule*[right=Group-Done]{
  g.\texttt{done} = 1 \text{ in } \sigma
}{
  \langle g_\text{body}, \sigma \rangle \rightarrow \langle \texttt{done}, \sigma \rangle
}
\end{mathpar}

To handle the latency-insensitive nature of Calyx, we can introduce a rule for delays:

\begin{mathpar}
\inferrule*[right=Delay]{
  \text{true}
}{
  \langle \delta, \sigma \rangle \rightarrow \langle \texttt{done}, \sigma \rangle
}
\end{mathpar}

This rule allows for arbitrary delays between operations, reflecting the latency-insensitive design of Calyx.

These operational semantics reflect the "Explicit Synchronization" approach in the following ways:

\begin{itemize}
  \item The Par-Left and Par-Right rules allow for non-deterministic interleaving of parallel executions, capturing the idea that parallel arms can execute in any order.
  \item The Group-Start and Group-Done rules explicitly model the activation and completion of groups, similar to the proposed synchronization points.
  \item The Delay rule allows for modeling of latency-insensitive designs, where the exact timing of operations is not fixed.
\end{itemize}

% \paragraph{Explanation of Rules}

% The operational semantics rules capture the essence of Calyx's parallel execution model:

% \begin{itemize}
%   \item \textbf{Par-Left} and \textbf{Par-Right}: These rules allow either the left or right component of a par construct to make progress independently. This captures the non-deterministic nature of parallel execution in hardware, where components may progress at different rates.
  
%   \item \textbf{Par-Done}: This rule specifies that a par construct completes only when both of its components have finished execution. This ensures proper synchronization at the end of parallel blocks.
  
%   \item \textbf{Group-Start}: This rule models the activation of a group by setting its 'go' signal. It transitions the group from its initial state to its body, representing the start of the group's execution.
  
%   \item \textbf{Group-Done}: This rule captures the completion of a group when its 'done' signal is set. It transitions the group's body to a done state, signifying the end of the group's execution.
  
%   \item \textbf{Delay}: This rule allows for arbitrary delays in execution, reflecting the latency-insensitive nature of Calyx. It can transition to a done state at any point, modeling variable-length operations or communication delays.
% \end{itemize}

% These rules provide a flexible framework for reasoning about Calyx programs. They allow for the modeling of concurrent execution (via Par-Left and Par-Right), explicit synchronization (via Group-Start and Group-Done), and latency-insensitive designs (via Delay). The non-determinism inherent in hardware execution is captured by the ability to apply Par-Left or Par-Right in any order, while still ensuring overall correctness through the Par-Done rule.

% \section{Relation to KA Formulas}

% The operational semantics can be related back to the KA formulas as follows:

% \begin{align*}
% \texttt{par}\{p_1, p_2\} &\equiv p_1 \times p_2 \\
% g &\equiv g.\texttt{go} \cdot g_\text{body} \cdot g.\texttt{done} \\
% \delta &\equiv 1 + \delta \cdot \delta \quad \text{(arbitrary delay)}
% \end{align*}

% Where $\times$ is the synchronous product from SKA, and $\cdot$ represents sequential composition. The delay operator $\delta$ is defined recursively to allow for arbitrary delays.

% This formulation allows us to reason about Calyx programs using both operational semantics and algebraic laws, providing a rich framework for analysis and optimization.

% \subsection{Optimizations and Analysis}
% \begin{itemize}
%     \item Develop algebraic optimizations for \texttt{par} constructs using SKA laws
%     \item Implement static analysis techniques to detect potential data races
%     \item Explore scheduling optimizations based on the formal model
% \end{itemize}

% \subsection{Integration with Existing Calyx Infrastructure}
% \begin{itemize}
%     \item Extend the Calyx IR to support the refined \texttt{par} semantics 
%     \item Modify the Calyx compiler to use the new semantics and optimizations
%     \item Ensure backwards compatibility with existing Calyx programs
% \end{itemize}

% \subsection{Evaluation}
% \begin{itemize}
%     \item Benchmark suite to test correctness and performance
%     \item Case studies demonstrating improved reasoning about parallel Calyx programs
%     \item Comparison with existing approaches (e.g., HLS tools)
% \end{itemize}

\section{Technical Questions}

\subsection{Questions for Synchronous Kleene Algebra Authors}
\begin{enumerate}
    \item How can the SKA framework be extended to handle latency-insensitive designs?
    \item Is it possible to define a "loose" synchronous product that allows for non-deterministic execution order while preserving essential properties?
    \item How can we incorporate the concept of clock cycles or timing into the SKA framework to better model hardware behavior?
    \item Are there any established techniques for handling state and side effects in SKA that could be applicable to Calyx's stateful components?
\end{enumerate}

\subsection{Questions for Calyx Semantics}
\begin{enumerate}
    \item How can we formally define the semantics of Calyx's groups and their interaction with the \texttt{par} construct?
    \item What is the precise meaning of the "done" signal in Calyx, and how should it be incorporated into the formal semantics?
    \item How should we handle potential conflicts between parallel executions of groups that modify the same state?
    \item What is the desired level of determinism for Calyx's \texttt{par} construct, and how can this be formally specified?
\end{enumerate}

\section{KA Formulas in Calyx Context}

To illustrate how KA formulas would look in the context of a Calyx program, consider the following Calyx snippet:

\begin{lstlisting}[language=Calyx]
group g1 {
  x.in = a.out;
  g1[done] = x.done;
}

group g2 {
  y.in = b.out;
  g2[done] = y.done;
}

control {
  par { g1; g2 }
}
\end{lstlisting}

We can represent this using KA formulas as follows:

\begin{align*}
g1 &\coloneqq (x.in \leftarrow a.out) \cdot (g1[done] \leftarrow x.done) \\
g2 &\coloneqq (y.in \leftarrow b.out) \cdot (g2[done] \leftarrow y.done) \\
par(g1, g2) &\coloneqq g1 \times g2
\end{align*}

Where $\times$ represents the synchronous product from SKA. We can then define properties such as:

\begin{align*}
par(g1, g2) &= par(g2, g1) \quad \text{(commutativity)} \\
par(g1, par(g2, g3)) &= par(par(g1, g2), g3) \quad \text{(associativity)}
\end{align*}

To handle latency-insensitive designs, we might introduce a delay operator $\delta$ and modify our formulas:

\begin{align*}
g1 &\coloneqq (x.in \leftarrow a.out) \cdot \delta \cdot (g1[done] \leftarrow x.done) \\
g2 &\coloneqq (y.in \leftarrow b.out) \cdot \delta \cdot (g2[done] \leftarrow y.done)
\end{align*}

This allows for a variable number of cycles between the input assignment and the done signal.

\end{document}